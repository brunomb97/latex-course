\documentclass[../slides.tex]{subfiles}

\begin{document}
    \begin{frame}
        \tableofcontents[sections=\value{section}]
    \end{frame}
        
    \begin{frame}[fragile]{Beamers}
        Beamer es una clase de documento para realizar presentaciones.

            \begin{verbatim}
\documentclass{beamer}
            \end{verbatim}

        La estructura básica de un beamer corresponderán a slides, de forma que denotaremos \texttt{frame} a cada diapositiva de la presentación.
        
        \begin{verbatim}
\ begin{frame}{Titulo de la diapositiva}
    Aqui va el texto de la diapositiva.
\ end{frame}
        \end{verbatim}

        En general, siguen las mismas reglas vistas anteriormente. Beamer entrega más herramientas para utilizar.
    \end{frame}

    \subsection{Múltiples columnas en diapositivas}

    \begin{frame}[fragile]{Múltiples columnas}
        Se puede utilizar múltiples columnas con el ambiente \texttt{columns}
        \begin{verbatim}
\begin{columns}
    \column{0.4 \linewidth}
        Este es el texto de la columna 1.
            
        Puede contener varios párrafos.
    \column{0.6 \linewidth}
        Este es el texto de la columna 2.
            
        Esta es una columna más ancha que la otra.
\end{columns}
        \end{verbatim}
    Output:
    
        \begin{columns}
            \column{0.4 \linewidth}
                Este es el texto de la columna 1.

                Puede contener varios párrafos.
            \column{0.6 \linewidth}
                Este es el texto de la columna 2.

                Esta es una columna más ancha que la otra.
        \end{columns}
     \end{frame}

     \begin{frame}{Múltiples columnas}
        \begin{block}{Nota}
            Tiene sentido establecer el largo de la columna de forma que dependa del largo de la linea. En este caso, se estableció que la primera columna posea el 40\% del largo de la linea y la segunda columna el 60\%.
        \end{block}
     \end{frame}
\end{document}