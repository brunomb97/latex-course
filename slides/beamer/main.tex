\documentclass[../slides.tex]{subfiles}

\begin{document}
    \begin{frame}
        \tableofcontents[sections=\value{section}]
    \end{frame}
    
    \subsection{Estructura}
    
    \begin{frame}[fragile]{Beamers}
        Beamer es una clase de documento para realizar presentaciones.

            \begin{verbatim}
\documentclass{beamer}
            \end{verbatim}

        La estructura básica de un beamer corresponderán a slides, de forma que denotaremos \texttt{frame} a cada diapositiva de la presentación.
        
        \begin{verbatim}
\\begin{frame}{Titulo de la diapositiva}
    Aqui va el texto de la diapositiva.
\\end{frame}
        \end{verbatim}

        En general, siguen las mismas reglas vistas anteriormente. Beamer entrega más herramientas para utilizar.
    \end{frame}

    
    \begin{frame}[fragile]{Más estructura}
        Se pueden también crear título e índices para las presentaciones.
        \begin{verbatim}
\\begin{frame}
    \maketitle
\\end{frame}

\\begin{frame}{Indice}
    \tableofcontents
\\end{frame}
        \end{verbatim}
        
        \begin{block}{Nota}
        	Existen más opciones para agregar al título que los vistos anteriormente. En este caso, también podemos agregar:
        \end{block}
        \begin{verbatim}
\subtitle{Aquí va un texto corto}
\logo{\includegraphics{utfsm}}
        \end{verbatim}

    \end{frame}
        
    
    \subsection{Múltiples columnas en diapositivas}

    \begin{frame}[fragile]{Múltiples columnas}
        Se puede utilizar múltiples columnas con el ambiente \texttt{columns}
        \begin{verbatim}
\begin{columns}
    \column{0.4 \linewidth}
        Este es el texto de la columna 1.
            
        Puede contener varios párrafos.
    \column{0.6 \linewidth}
        Este es el texto de la columna 2.
            
        Esta es una columna más ancha que la otra.
\end{columns}
        \end{verbatim}
    Output:
    
        \begin{columns}
            \column{0.4 \linewidth}
                Este es el texto de la columna 1.

                Puede contener varios párrafos.
            \column{0.6 \linewidth}
                Este es el texto de la columna 2.

                Esta es una columna más ancha que la otra.
        \end{columns}
     \end{frame}

     \begin{frame}{Múltiples columnas}
        \begin{block}{Nota}
            Tiene sentido establecer el largo de la columna de forma que dependa del largo de la linea. En este caso, se estableció que la primera columna posea el 40\% del largo de la linea y la segunda columna el 60\%.\\[\baselineskip]
            
            No necesariamente debe cubrir toda la linea.
        \end{block}
     \end{frame}
     
     \begin{frame}[fragile]{Múltiples columnas}
        \begin{verbatim}
    Este texto se encuentra arriba.
        \vspace{1em}
\begin{columns}
    \column{0.4 \linewidth}
        Este es el texto de la columna 1.
            
        Puede contener varios párrafos.
    \column{0.4 \linewidth}
        Este es el texto de la columna 2.
            
        Esta es una columna más ancha que la otra.
\end{columns}
        \vspace{1em}
    Este texto se encuentra abajo.
        \end{verbatim}
     \end{frame}
     
     \begin{frame}{Múltiples columnas}
    Este texto se encuentra arriba de las columnas.
    \vspace{1em}
\begin{columns}
    \column{0.4 \linewidth}
        Este es el texto de la columna 1.
            
        Puede contener varios párrafos.
    \column{0.4 \linewidth}
        Este es el texto de la columna 2.
            
        Esta es una columna más ancha que la otra.
\end{columns}
    \vspace{1em}
    Este texto se encuentra abajo de las columnas.
     \end{frame}

    \subsection{Pausas}
    
    \begin{frame}[fragile]{Pausas}
    	Una diapositiva se puede separar en varias a medida que se agregan pausas: hacer aparecer una linea nueva en la slide cada vez que se avanza en la presentación.
        \begin{verbatim}
\\begin{frame}
    Este texto aparecerá en una primera instancia.
    
    \pause
    
    Este texto aparecerá en una segunda instancia.
\\end{frame}
        \end{verbatim}

    \end{frame}

        \subsection{}
    
    \begin{frame}[fragile]{Capas especificas}
    	Además de las pausas, se pueden especificar las slides en las que aparecerán, utilizando \texttt{<>}.
        \begin{verbatim}
\\begin{frame}
    \begin{itemize}
        \item<2> Esta solo aparecerá en la slide 2
        \item<1-4> Esta aparecerá desde la slides 1 
        hasta la 4, incluyéndolas.
        \item<3-> Esta aparecerá desde la slide 3 en adelante.
        \item<1,3-4,6-> Esta aparecerá en la slide 1, 
        desde la 3 hasta la 4 y desde la 6 en adelante.
    \end{itemize}
\\end{frame}
        \end{verbatim}
    \end{frame}
    
    \subsection{Bloques}
    \begin{frame}[fragile]{Bloques}
        \begin{verbatim}
\\begin{frame}
    \begin{block}{Titulo del bloque}
        Este texto se encuentra dentro del bloque.
    \end{block}
    \begin{alertblock}{Bloque de alerta}
        Este texto se encuentra dentro de un bloque de alerta.
    \end{block}
\\end{frame}
        \end{verbatim}
        \begin{block}{Titulo del bloque}
            Este texto se encuentra dentro del bloque.
        \end{block}
        \begin{alertblock}{Bloque de alerta}
            Este texto se encuentra dentro de un bloque de alerta.
        \end{alertblock}
    \end{frame}
    
    \subsection{Temas}
    \begin{frame}[fragile]{Temas}
        Para seleccionar un tema, se deben agregar los comandos \texttt{usetheme} y \texttt{usecolortheme} en el preámbulo.
        \begin{verbatim}
        	\usetheme{Frankfurt}
        \end{verbatim}
        
        \begin{block}{Nota}
        	Los temas de beamer poseen nombres de ciudades; los temas de color poseen nombres de animales.
        \end{block}

        Existen muchos temas disponibles para beamer. Se puede encontrar un listado de estas en
            \begin{itemize}
            	\item \href{https://www.overleaf.com/learn/latex/Beamer%23Creating_a_table_of_contents#Reference_guide}{Overleaf: guía de referencia}
            	\item \href{https://latex-beamer.com/tutorials/beamer-themes/}{Lista completa de Temas de Beamer}
            \end{itemize}

    \end{frame}
    
\end{document}
