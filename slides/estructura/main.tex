\documentclass[../slides.tex]{subfiles}

\begin{document}
    
    \begin{frame}[fragile]

    Un documento en \LaTeX{} se compone de dos principales segmentos: el preámbulo y el cuerpo del documento (denotado por \texttt{document}).\\
    \vspace{1em}
        $\left. \begin{minipage}{0.8\linewidth}
            \begin{verbatim}
\documentclass[]{article}

\usepackage[margin=1.25in]{geometry}
\usepackage{amsmath, amssymb}
\usepackage{graphicx}

\title{Curso de \LaTeX{}}
\author{Bruno Martinez Barrera}
\date{25 de octubre de 2022}
            \end{verbatim}
        \end{minipage} \right\rbrace$ Preámbulo
        
        $\left. \begin{minipage}{0.8\linewidth}
            \begin{verbatim}
\begin{document}
    Aquí va el texto.

    \[ f(x) = \int_{0}^{x} t^{2} dt\] %Integral

    \includegraphics[scale=0.5]{image.png}
\end{document}
            \end{verbatim}
        \end{minipage} \right\rbrace$ Documento
    \end{frame}

    \begin{frame}
        \tableofcontents[sections=\value{section}]
    \end{frame}
    \subsection{Preámbulo}

    \begin{frame}{\texttt{documentClass}}
        \subsection{\texttt{documentclass}}
        El preámbulo debe iniciar con el comando \texttt{documentclass}, que especifica la estructura general del documento. Este comando se estructura de la siguiente forma:
        \[ \text{\texttt{\textbackslash documentclass}}\underbrace{\text{\texttt{[letterpage,12pt]}}}_{\text{opcional}}\underbrace{\text{\texttt{\{article\}}}}_{\text{tipo de documento}} \]
    \end{frame}

    \begin{frame}{\texttt{documentClass}}
        \[ \text{\texttt{\textbackslash documentclass}}\underbrace{\text{\texttt{[letterpage,12pt]}}}_{\text{opcional}}\underbrace{\text{\texttt{\{article\}}}}_{\text{tipo de documento}} \]
            \begin{enumerate}
                \item \textbf{Tipo de documento}: 
                    \begin{itemize}
                        \item \texttt{article}: Estructura general para escribir articulos, ya sea publicaciones cientificas, documentación para programas y, en general, cualquier tipo de documento corto que no requiera una división compleja de partes y capítulos.
                        \item \texttt{report}: Tipos de articulos largos, que contengan una cantidad mayor de capítulos, como libros pequeños y tesis.
                        \item \texttt{book}: Usado para libros y trabajos extensos.
                        \item \texttt{letter}: Para cartas.
                        
                    \end{itemize}
                \end{enumerate}
    \end{frame}

    \begin{frame}
        \[ \text{\texttt{\textbackslash documentclass}}\underbrace{\text{\texttt{[letterpage,12pt]}}}_{\text{opcional}}\underbrace{\text{\texttt{\{article\}}}}_{\text{tipo de documento}} \]
            \begin{enumerate}
                \setcounter{enumi}{1}
                \item Opcional:
                    \begin{itemize}
                        \item Tamaño de papel:
                        \item Tamaño de fuente:
                    \end{itemize}
            \end{enumerate}
        
    \end{frame}
    
    \subsection{\texttt{usepackage}}
    
    \begin{frame}{\texttt{usepackage}}
        
        Luego, suelen agregarse distintos paquetes. Los siguientes
            \begin{itemize}
                \item \texttt{geometry}: Permite editar el diseño de página. En particular, nos permitirá modificar los márgenes.
                \item \texttt{amsmath}: Contiene herramientas muy útiles para el modo matemática. Es mantenido por American Mathematical Society, y entrega los comandos más usados por aquellos que escriben en LaTeX.
                \item \texttt{amssymb}: Contiene herramientas muy útiles para el modo matemática. En particular, simbolos matemáticos.
                \item \texttt{graphicx}: Permite importar y manipular de imagenes.
            \end{itemize}
    \end{frame}

    \subsection{Título, autor y fecha}
    \begin{frame}
        \begin{enumerate}
            \item \texttt{\textbackslash title}: El título del documento.
            \item \texttt{\textbackslash author}: Para indicar el nombre de los autores.
            \item \texttt{\textbackslash date}: puedes ingresar la fecha manualmente, o bien utilizar el comando \texttt{\textbackslash today} para indicar la fecha actual de la última vez que se compiló el archivo.
            \item \texttt{\textbackslash institute}: universidad o institución asociada.
        \end{enumerate}
    \end{frame}
    \subsection{Ambiente \texttt{document}}
    
    \begin{frame}{Ambiente \texttt{document}}
        En el ambiente \texttt{document} es donde se presenta el cuerpo: el contenido actual que se desea mostrar.
    \end{frame}

    \subsection{Comentarios}

    \begin{frame}[fragile]{Comentarios}
        De alguna forma, en \LaTeX{} se programa código. Por tanto, puede ser útil \emph{incluir comentarios} dentro de los documentos.
        
        Ejemplos para utilizar comentarios:
        
            \begin{itemize}
                \item Para indicar notas de "tareas pendientes".
                \item Comentarios que expliquen el tema.
            \end{itemize}
            
        Esto se hace de forma de escribir \texttt{\%} y el comentario deseado. Agregar texto después de un \% no aparecerá en el documento final.
        
        \begin{verbatim}
Este texto si aparecerá en el documento. % Sin embargo,
                                            esto permanecerá
                                            como comentario
                                            sólo en el código.
        \end{verbatim}

    \end{frame}
    
    \subsection{Unidades de medida}

    \begin{frame}[fragile]{Unidades de medida}
        Se suelen utilizar en algunos comandos donde un párametro corresponde a la medida de algo. Por ejemplo: márgenes, anchos de columnas, etc.
        \begin{center}
            \begin{tabular}{ll}
                in & pulgadas\\
                cm & centimetros\\
                mm & milimetros\\
                pt & puntos\\
                em & el ancho de una m\\
                ex & el alto de una x
            \end{tabular}
        \end{center}
    \end{frame}
\end{document}
