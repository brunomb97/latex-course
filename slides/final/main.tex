\documentclass[../slides.tex]{subfiles}

\begin{document}

    \begin{frame}{Estudios para más adelante}
        \begin{enumerate}
        	\item Estructuras de poster, minipaginas, documentos de dos columnas.
        	\item Estudio a profundidad de secciones.
        	\item Estilización, personalización de márgenes de página, pies de página, enumeración de páginas y cabeceras. Recomiendo utilizar el paquete \texttt{fancyhdr} para esto.
        	\item El uso del paquete \texttt{bibtex} para añadir bibliografia a articulos cientificos.
        	\item Referencias a secciones y ecuaciones con \texttt{label} y \texttt{ref}.
        	\item Utilizar múltiples archivos de \LaTeX{}, para dividir secciones grandes. Recomiendo utilizar \texttt{subfiles}
        \end{enumerate}
    \end{frame}
    
    \subsection{Otras herramientas útiles}
    \begin{frame}{Otras herramientas útiles}
       \begin{enumerate}
            \item \texttt{verbatim}: Incluido en \LaTeX{}. Permite escribir códigos con fuente de maquina de escribir.
            \item \texttt{tikz}: El paquete más potente para gráficos en general.
            \item \texttt{minted}: Paquete que provee ambientes de highlighting de códigos, según el lenguaje.
            \item \texttt{exam}: Es un tipo de documento (y por tanto, va en \texttt{documentclass}). Permite crear evaluaciones y preguntas.
            \item \texttt{pgfplots}: Paquete basado en \texttt{TikZ}. Es una herramienta de visualización potente para graficos técnicos y cientificos.
            \item \texttt{amsthm}: Paquete de ambientes para definiciones, teoremas y demostraciones.
            \item \texttt{algpseudocode}: Paquete de ambientes para pseudo códigos.
            \item \texttt{chemfig}: Paquete de ambientes para fórmulas químicas.
            \item \texttt{tikz-feynman}: Paquete de ambientes para diagramas de Feynman.
            \item \texttt{circuitikz}: Paquete de ambientes para circuitos eléctricos y eléctronicos.
       \end{enumerate}
    \end{frame}
    
    \begin{frame}{Consejos generales}
    	\begin{enumerate}   
    		\item Ser ordenado. Considera que, de vez en cuando, puede tocarte trabajo en equipo utilizando \LaTeX{}.
            \item Es una buena practica indentar dentro del código de \LaTeX{}, para mantener el orden.
            \item Compilar de manera continua, para detectar el error de forma rápida y no ir arrastrando.
            \item Utilizar foros e internet en general ante dudas sobre cómo escribir o mostrar algo. La comunidad de \LaTeX{} es enorme y muy bien documentada. (Generalmente en inglés).
    	\end{enumerate}

    \end{frame}

\end{document}
