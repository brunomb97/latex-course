\documentclass[../slides.tex]{subfiles}

\begin{document}

    \begin{frame}
        \tableofcontents[sections=\value{section}]
    \end{frame}

    \subsection{Modo matemática}

    \begin{frame}[fragile]{Modo matemática}
        \begin{columns}[t]
            \column{0.5\linewidth}
                \textbf{In-line}: se utiliza para escribir fórmulas dentro del párrafo. Se utiliza \texttt{\$math\$}.
            \column{0.5\linewidth}
                \textbf{Display}: se utiliza para escribir fórmulas fuera del párrafo, y por tanto, en una linea separada. Se puede utilizar tanto \texttt{\$\$math\$\$} como \texttt{\textbackslash[math\textbackslash]}.
        \end{columns}
        \begin{verbatim}
Como $f(x) = x^3$, entonces
    $$ \int f(x) dx = \frac{x^4}{4} $$.
De esta forma,
    \[ \int_{0}^{1} f(x) dx = 
    \frac{1^4}{4} - \frac{0^4}{4} = \frac{1}{4} \]
        \end{verbatim}
Como $f(x) = x^3$, entonces
    $$ \int f(x) dx = \frac{x^4}{4} $$.
De esta forma,
    \[ \int_{0}^{1} f(x) dx = 
    \frac{1^4}{4} - \frac{0^4}{4} = \frac{1}{4} \]
    \end{frame}

    \subsection{Notación básica}
    
    \begin{frame}{Notación básica}
        \begin{enumerate}
            \item Aritmética
                \begin{center}
                \begin{tabular}{|c|c|c|}
                \hline
                    Suma & \texttt{1 + 2 = 3} & $\quad\quad1 + 2 = 3\quad\quad$\\
                \hline
                    Resta & \texttt{8 - 1 = 7} & $8 - 1 = 7$\\
                \hline
                    Multiplicación & \texttt{3 \textbackslash cdot 4 = 12} & $3 \cdot 4 = 12$\\
                \hline
                    Multiplicación & \texttt{3 \textbackslash times 4 = 12} & $3 \times 4 = 12$\\
                \hline
                    División & \texttt{25 \textbackslash div 5 = 5} & $25 \div 5 = 5$\\
                \hline
                \end{tabular}
                \end{center}
            \item Fracciones
                \begin{center}
                \begin{tabular}{|c|c|c|}
                \hline
                    & & \\
                    Fracción & \texttt \textbackslash frac\{12\}\{48\} & $\quad\quad\dfrac{1}{4}\quad\quad$ \\
                    & & \\
                \hline
                \end{tabular}
                \end{center}
        \end{enumerate}

    \end{frame}

    \begin{frame}{Notación básica}
        \begin{block}{Nota}
            \begin{itemize}
                \item Es buena práctica utilizar fracciones en vez de divisiones.
                \item Utilizar \texttt{frac} en modo in-line entregará una versión más compacta de esta. En caso de necesitar la versión ampliada, se puede utilizar \texttt{dfrac}. En el caso de estar en modo display, no habrá diferencia.
            \end{itemize}
        \end{block}
        \begin{center}
           \begin{tabular}{|c|c|c|}
             \hline
                 & Modo in-line & Modo display\\
            \hline
                & & \\
                 \texttt \textbackslash frac\{12\}\{48\} & $\quad\quad\frac{1}{4}\quad\quad$ & $\quad\quad\dfrac{1}{4}\quad\quad$ \\
                        & & \\
                \texttt \textbackslash dfrac\{12\}\{48\} & $\quad\quad\dfrac{1}{4}\quad\quad$ & $\quad\quad\dfrac{1}{4}\quad\quad$ \\
                        & & \\
              \hline
          \end{tabular}
        \end{center}
    \end{frame}

    \subsection{Superlineado y sublineado}

    \begin{frame}[fragile]{Superlineado y sublineado}
        \begin{columns}[t]
            \column{0.5\linewidth}
                \textbf{Superlineado}: 

                Se utiliza \texttt{\^} para escribir en superlineado. Por ejemplo, exponentes.

                \begin{verbatim}
Se tiene entonces que 
$e^{i\pi} = -1$.
                \end{verbatim}
Se tiene entonces que $e^{i\pi} = -1$.
            \column{0.5\linewidth}
                \textbf{Sublineado}: 

                Se utiliza \texttt{\_} para escribir en sublineado. Por ejemplo, índices.

                \begin{verbatim}
Por tanto, existe una
subsucesión $x_{k_{j}}$.
                \end{verbatim}
Por tanto, existe una subsucesión $x_{k_{j}}$.
        \end{columns}
    \end{frame}
    
    \begin{frame}{Notación para cálculo}
        \begin{enumerate}
            \item Limite
            
            \begin{center}
                \begin{tabular}{|c|c|}
                    \hline
                        & \\
                        \begin{tabular}{@{}c@{}} \texttt{\textbackslash lim\_\{x\textbackslash{}to\textbackslash{}infty\}  \textbackslash frac\{x\^{}2\}\{2x\^{}2+1\}}\\\texttt{ = \textbackslash frac 1 2} \end{tabular} 
                        & $\displaystyle \quad\lim_{x \to \infty} \frac{x^2}{2x^2+1} = \frac 1 2\quad$ \\
                        & \\
                    \hline
                \end{tabular}
            \end{center}

            \item Sumatoria
            
            \begin{center}
                \begin{tabular}{|c|c|}
                    \hline
                        & \\
                        
                        \texttt {\textbackslash sum\_\{i = 1\}\^{}n x = \textbackslash frac\{x(x+1)\}\{2}\} & $\displaystyle \quad\sum_{i = 1}^n x = \dfrac{x(x+1)}{2} \quad$ \\
                        & \\
                    \hline
                \end{tabular}
            \end{center}
        \end{enumerate}
    \end{frame}
    

    
    \subsection{Paréntesis}

    \begin{frame}{Paréntesis}
        %Teoria de conjuntos
    \end{frame}
    
    \subsection{Texto en modo matemática}
    
    \begin{frame}{Texto en modo matemática}
        %Teoria de conjuntos
    \end{frame}        
    
    \subsection{Letras griegas}

    \begin{frame}{Letras griegas}
        %Teoria de conjuntos
    \end{frame} 
    
    \subsection{Notación para calculo}

    \begin{frame}{Texto en modo matemática}
        %Teoria de conjuntos
    \end{frame}
        \begin{enumerate}
        \item{Funciones}
        \item{Limites}
        \item{Sumas}
            \item{Comando ``substack''}
        \item{Integrales}
        \item{Derivadas}
        \item{Vectores}
            Ejemplos: Ley Faraday
    
        \item{Radicales}
        
        \item{Logica y cuantificadores}
        
        \item{Relaciones de conjuntos}
        
        \item{Combinatoria} 
        \end{enumerate}
\end{document}