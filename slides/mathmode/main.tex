\documentclass[../slides.tex]{subfiles}

\begin{document}

    \begin{frame}
        \tableofcontents[sections=\value{section}]
    \end{frame}

    \subsection{Modo matemática y algunos detalles}

    \begin{frame}{Modo matemática y algunos detalles}
    \end{frame}
    
    \subsection{Notación básica}
        \begin{itemize}
            \item Aritmética
            \item Fracciones
        \end{itemize}{Aritmetica}
        
    \subsection{Superlineado y sublineado}

    \begin{frame}{Superlineado y sublineado}
    \end{frame}
    
    \subsection{Paréntesis}

    \begin{frame}{Paréntesis}
        %Teoria de conjuntos
    \end{frame}
    
    \subsection{Texto en modo matemática}
    
    \begin{frame}{Texto en modo matemática}
        %Teoria de conjuntos
    \end{frame}        
    
    \subsection{Letras griegas}

    \begin{frame}{Letras griegas}
        %Teoria de conjuntos
    \end{frame} 
    
    \subsection{Notación para calculo}

    \begin{frame}{Texto en modo matemática}
        %Teoria de conjuntos
    \end{frame}
        \begin{enumerate}
        \item{Funciones}
        \item{Limites}
        \item{Sumas}
            \item{Comando ``substack''}
        \item{Integrales}
        \item{Derivadas}
        \item{Vectores}
            Ejemplos: Ley Faraday
    
        \item{Radicales}
        
        \item{Logica y cuantificadores}
        
        \item{Relaciones de conjuntos}
        
        \item{Combinatoria} 
        \end{enumerate}
\end{document}