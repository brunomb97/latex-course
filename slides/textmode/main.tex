\documentclass[../slides.tex]{subfiles}

\begin{document}

    \begin{frame}
        \tableofcontents[sections=\value{section}]
    \end{frame}
    
    \subsection{Tamaños de fuente}
    \begin{frame}{Tamaños de fuente}
        \begin{enumerate}
            \item \textbf{En todo el texto}: Se define en \texttt{\textbackslash documentclass[]\{\}}

            \item \textbf{En porciones del texto}:
                Comandos:
                \begin{enumerate}
                    \item \texttt{\textbackslash tiny}:
                    \item \texttt{\textbackslash scriptsize}:
                    \item \texttt{\textbackslash footnotesize}:
                    \item \texttt{\textbackslash normalsize}:
                    \item \texttt{\textbackslash large}:
                    \item \texttt{\textbackslash Large}:
                    \item \texttt{\textbackslash LARGE}:
                    \item \texttt{\textbackslash huge}:
                    \item \texttt{\textbackslash Huge}:
                \end{enumerate}
                
        \end{enumerate}
    \end{frame}
        
    \subsection{Estilos de fuente}
    \begin{frame}{Estilos de fuente}
        En particular, tenemos tres herramientas usuales para dar estilo al texto presentado.
            \begin{enumerate}
                \item Negrita: \texttt{\textbackslash textbf} Este es un texto en \textbf{negrita}.
                    
                \item Cursiva: \texttt{\textbackslash textit}
                \item Subrayado: \texttt{\textbackslash underline}
            \end{enumerate}
    \end{frame}

    \subsection{Correcto uso de brackets}
    \begin{frame}{Correcto uso de brackets}
    
    \end{frame}
    
    \subsection{Listas}
    \begin{frame}[fragile]{Listas}
        \begin{enumerate}
            \item Lista enumerada:
                \begin{verbatim}
\begin{enumerate}
    \item Este es el primer item.
    \item Este es el segundo item.
\end{enumerate}
                \end{verbatim}
            \item Lista no enumerada:
                \begin{verbatim}
\begin{itemize}
    \item Este es el primer item.
    \item Este es el segundo item.
\end{itemize}
                \end{verbatim}
        \end{enumerate}
    \end{frame}
    
    \subsection{Alineamiento de texto}
    \begin{frame}[fragile]{Alineamiento de texto}
        \begin{columns}
            \column{0.5\linewidth}
                \begin{verbatim}
Este texto no está centrado.
                \end{verbatim}
Este texto no está alineado.
            \column{0.5\linewidth}
                Texto centrado
                    \begin{verbatim}
\begin{center}
Este texto está centrado.
\end{center}
                    \end{verbatim}
\begin{center}
Este texto está centrado.
\end{center}
        \end{columns}
    \end{frame}

        \begin{frame}[fragile]{Alineamiento de texto}
        \begin{columns}
            \column{0.5\linewidth}
                Texto a la izquierda
                    \begin{verbatim}
\begin{flushleft}
Este texto está alineado a
la izquierda.
\end{flushleft}
                    \end{verbatim}
\begin{flushleft}
Este texto está alineado a la izquierda.
\end{flushleft}
            \column{0.5\linewidth}
                Texto a la derecha
                    \begin{verbatim}
\begin{flushright}
Este texto está alineado
a la derecha.
\end{flushright}
                    \end{verbatim}
\begin{flushright}
Este texto está alineado a la derecha.
\end{flushright}
        \end{columns}
    \end{frame}

    \subsection{Saltos de linea}

    \begin{frame}[fragile]{Saltos de linea}
        Los saltos de linea se pueden realizar como
            \begin{verbatim}
Esta es la primera linea.\\
Esta es la segunda linea.\\[\baselineskip]
Esta es la tercera linea.\\[2\baselineskip]
            \end{verbatim}
Esta es la primera linea.\\
Esta es la segunda linea.\\[\baselineskip]
Esta es la tercera linea.\\[2\baselineskip]
    \end{frame}

    \subsection{Forzar nueva página}
    \begin{frame}{Forzar nueva página}
        Dos comandos:
            \begin{enumerate}
                \item \texttt{\textbackslash newpage}
                \item \texttt{\textbackslash clearpage}
            \end{enumerate}
        Ambos comandos se utilizan para saltar una nueva página. Poseen unas diferencias, las cuales se verán más adelante.
    \end{frame}
    
    \subsection{Otras utilidades para estudiar}
    
    \begin{frame}{Otras utilidades para estudiar}
       % \subsection{Ambientes para definiciones, teoremas y demostraciones}}
    \end{frame}
    
\end{document}