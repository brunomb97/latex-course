\documentclass[../notes.tex]{subfiles}
\begin{document}

    En esta sección aprenderemos a

    Un documento en \LaTeX{} se compone de dos principales segmentos: el preámbulo y el ambiente general del documento (denotado por \texttt{document}).\\
    
        $\left. \begin{minipage}{0.65\linewidth}
            \begin{verbatim}
\documentclass[]{article}

\usepackage[margin=1.25in]{geometry}
\usepackage{amsmath, amssymb}
\usepackage{graphicx}
            \end{verbatim}
        \end{minipage} \right\rbrace$ Preámbulo
        
        $\left. \begin{minipage}{0.65\linewidth}
            \begin{verbatim}
\begin{document}
    Aquí va el texto.

    \[ f(x) = \int_{0}^{x} t^{2} dt\]

    \includegraphics[scale=0.5]{image.png}
\end{document}
            \end{verbatim}
        \end{minipage} \right\rbrace$ Ambiente "\texttt{document}"

    \section{Preámbulo}
        \subsection{\texttt{documentclass}}
        El preámbulo debe iniciar con el comando \texttt{documentclass}, que especifica la estructura general del documento. Este comando se estructura de la siguiente forma:
        \[ \text{\texttt{\textbackslash documentclass}}\underbrace{\text{\texttt{[letterpage,12pt]}}}_{\text{opcional}}\underbrace{\text{\texttt{\{article\}}}}_{\text{tipo de documento}} \]

        donde
            \begin{enumerate}
                \item \textbf{Tipo de documento}: 
                    \begin{itemize}
                        \item \texttt{article}: Estructura general para escribir articulos, ya sea publicaciones cientificas, documentación para programas y, en general, cualquier tipo de documento corto que no requiera una división compleja de partes y capítulos.
                        \item \texttt{report}: Tipos de articulos largos, que contengan una cantidad mayor de capítulos, como libros pequeños y tesis.
                        \item \texttt{book}: Usado para libros...
                        \item \texttt{letter}: Para cartas.
                        \item \texttt{beamer}: Utilizado para presentaciones en general. Requiere el paquete \texttt{beamer}.
                    \end{itemize}
                \item Opcional:
                    \begin{itemize}
                        \item Tamaño de papel:
                        \item Tamaño de fuente:
                    \end{itemize}
            \end{enumerate}

        \subsection{\texttt{usepackage}}
        Luego, suelen agregarse distintos paquetes. Los siguientes
            \begin{itemize}
                \item \texttt{geometry}:
                \item \texttt{amsmath}:
                \item \texttt{amssymb}:
                \item \texttt{graphicx}:
            \end{itemize}
    \section{Ambiente \texttt{document}}

        En el ambiente \texttt{document} es donde se presenta el contenido actual que se desea mostrar.
        
    \section{Tamaños de fuente}

        \subsection{En todo el texto}
        
            Para establecer un tamaño de fuente en todo el documento, se establece. Puede ser \texttt{10pt}, \texttt{11pt} y \texttt{12pt}.
            
        \subsection{En una porción de texto}
        
            Comandos:
                \begin{enumerate}
                    \item \texttt{\textbackslash tiny}:
                    \item \texttt{\textbackslash scriptsize}:
                    \item \texttt{\textbackslash footnotesize}:
                    \item \texttt{\textbackslash normalsize}:
                    \item \texttt{\textbackslash large}:
                    \item \texttt{\textbackslash Large}:
                    \item \texttt{\textbackslash LARGE}:
                    \item \texttt{\textbackslash huge}:
                    \item \texttt{\textbackslash Huge}:
                \end{enumerate}

        Alcance
    \section{Familia de fuentes}

    
    \section{Estilos de fuente}
    
    \section{Correcto uso de brackets}
    
    \section{Enfasís}
    
    
    
    \section{Alineamiento de texto}
    
    \section{Saltos de linea}
    
    \section{Indentación}
    
    \section{Importar imagenes}
    
    \section{Diseño y tamaño de la pagina}
    
        \subsection{Tipos de página}
        \subsection{Fancyhdr}
        \subsection{Números de página}
        \subsection{Forzar nueva página}
        
    \section{Definir nuevos comandos y funciones}
    
    \section{Enumeración y listados}
        \subsection{Customizar bullets}
        
    \section{Ambientes para definiciones, teoremas y demostraciones}
    
    \section{Definir nuevos ambientes}
    
\end{document}
