\documentclass[../notes.tex]{subfiles}
\begin{document}
    \section{Modo matemática y algunos detalles}
    
    \section{Notación básica}
        \subsection{Aritmetica}
        
        \subsection{Fracciones}
        
    \section{Superlineado y sublineado}
    
    \section{Paréntesis}
    
        \subsection{Teoria de conjuntos}
        
        
    \section{Texto en modo matemática}
    
    \section{Letras griegas}
    
    \section{Letras en blackbaord}
    
    \section{Notación para calculo}
        
        \subsection{Funciones}
        \subsection{Limites}
        \subsection{Sumas}
            \subsubsection{Comando ``substack''}
        \subsection{Integrales}
        \subsection{Derivadas}
        \subsection{Vectores}
            Ejemplos: Ley Faraday
    
        \subsection{Radicales}
        
        \subsection{Logica y cuantificadores}
        
        \subsection{Relaciones de conjuntos}
        
        \subsection{Combinatoria} 
        
\end{document}
