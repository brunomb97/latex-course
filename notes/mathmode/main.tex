\documentclass[../notes.tex]{subfiles}
\begin{document}
    \section{Modo matemática}
        Una de las mayores ventajas de {\LaTeX} es su capacidad de escribir fórmulas matemáticas complejas. Como ya se ha dicho anteriormente, {\LaTeX} separa la composición de textos en modo texto y modo matemática.
        
        En primer lugar, se puede presentar en una linea de texto, o bien desplegarse como una ecuación. Para el primer caso, se tiene
            
            \begin{Verbatim}[frame=single,fontsize=\small,label=Input]
La siguiente línea de $\int_2^3 x^2 \, dx = \frac{3^3}{3} - \frac{2^3}{3} 
= \frac{19}{3}.$
            \end{Verbatim}
            \fbox{
            \begin{minipage}{0.9665\linewidth}
            	La siguiente línea de $\int_2^3 x^2 \, dx = \frac{3^3}{3} - \frac{2^3}{3} = \frac{19}{3}.$
            \end{minipage}
            }
            
        Mientras que, en una versión extendida:
        
            \begin{Verbatim}[frame=single,fontsize=\small,label=Input]
La siguiente línea de
    \[\int_2^3 x^2 \, dx = \frac{3^3}{3} - \frac{2^3}{3} = \frac{19}{3} \]
            \end{Verbatim}
            \fbox{
            \begin{minipage}{0.9665\linewidth}
            	La siguiente línea de
                    \[\int_2^3 x^2 \, dx = \frac{3^3}{3} - \frac{2^3}{3} = \frac{19}{3} \]
            \end{minipage}
            }
        
        
    \section{Notación básica}
        \subsection{Aritmetica}
            \begin{Verbatim}[frame=single,fontsize=\small,label=Input]
\[ 1 + 2 - 3 = 0,\quad 3 \cdot 4 = 12,\quad 3 \times 4 = 12\]
            \end{Verbatim}
            \fbox{
            \begin{minipage}{0.9665\linewidth}
            	\[ 1 + 2 - 3 = 0,\quad 3 \cdot 4 = 12,\quad 3 \times 4 = 12\]
            \end{minipage}
            }
        \subsection{Fracciones}
        
    \section{Superlineado y sublineado}
    
    \section{Paréntesis}
    
        \subsection{Teoria de conjuntos}
        
        
    \section{Texto en modo matemática}
    
    \section{Letras griegas}
    
    \section{Letras en blackbaord}
    
    \section{Notación para calculo}
        
        \subsection{Funciones}
        \subsection{Limites}
        \subsection{Sumas}
            \subsubsection{Comando ``substack''}
        \subsection{Integrales}
        \subsection{Derivadas}
        \subsection{Vectores}
            Ejemplos: Ley Faraday
    
        \subsection{Radicales}
        
        \subsection{Logica y cuantificadores}
        
        \subsection{Relaciones de conjuntos}
        
        \subsection{Combinatoria} 
        
\end{document}
