\documentclass[letterpaper,12pt]{book}

% Selecciona el encoding del fondo
\usepackage[T1]{fontenc}

% Aporta soporte multilinguistico
\usepackage[spanish]{babel}
\usepackage[margin=1.25in]{geometry}

% Agregar metadatos al archivo
\usepackage[pdfusetitle]{hyperref}

% Agregar fancyvrb
\usepackage{fancyvrb}

% Agregar imágnes
\usepackage{graphicx}

% Agregar mathmode
\usepackage{amsmath}
\usepackage{amssymb}
\usepackage{amsthm}
\usepackage{esvect}

% Permitir escribir en múltiples archivos
\usepackage{subfiles}

\title{Curso de \LaTeX{}}
\author{Bruno Martinez Barrera}

\begin{document}
    
\maketitle
\tableofcontents

\part{LaTeX}
\chapter{Introducción}
	
    \subfile{introduccion/main.tex}

\chapter{Estructura de un archivo \LaTeX{}}
    
    \subfile{estructura/main.tex}
    
\chapter{Manipulación básica de textos}
    
    \subfile{textmode/main.tex}
    
\chapter{Manipulación básica en modo matemática}
    
    \subfile{mathmode/main.tex}
    
\chapter{Tablas y Arreglos}

    \subfile{tablas/main.tex}
    
\chapter{Beamer}

    \subfile{beamer/main.tex}
    
\chapter{Palabras finales}

    \subfile{conclusiones/main.tex}
    
\part{Tikz}
\chapter{El paquete ``tikz''}

    \subfile{tikz/main.tex}
    
\end{document}

